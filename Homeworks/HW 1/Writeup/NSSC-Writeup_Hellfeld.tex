\documentclass[11pt]{article}
\usepackage[left=0.9in,top=0.9in,bottom=0.9in,right=0.9in]{geometry}
\usepackage[english]{babel}
\usepackage{fancyhdr}
\usepackage{lastpage}
\usepackage{caption}
\usepackage{amsmath}
\usepackage{hyperref}
\usepackage{xcolor}
\usepackage{graphicx}
\usepackage{float}
\usepackage{textcomp}
\usepackage{amssymb}
\usepackage{mathrsfs}
\usepackage{soul}

%\pagestyle{fancy} 
%\fancyhead[LE,RO]{\today}
%\fancyhead[C]{NE 255}
%\fancyhead[LO,RE]{D. Hellfeld}
%\fancyfoot[C]{\thepage\ of \pageref{LastPage}}
%\renewcommand{\headrulewidth}{0.4pt}
%\renewcommand{\footrulewidth}{0.4pt}

\pagestyle{empty}

% Set up figure/table captions
\addto\captionsenglish{\renewcommand{\figurename}{Fig.}}
\addto\captionsenglish{\renewcommand{\tablename}{\small Table}}
\renewcommand{\thetable}{\Roman{table}}
\captionsetup[table]{labelfont = normal, labelsep=period, singlelinecheck=false}
\captionsetup[figure]{labelfont=normal, labelsep=period, singlelinecheck=false}

\setlength\parindent{0pt}

\begin{document}

\begin{centering}
\textbf{NSSC-LLNL Kickoff Meeting Writeup }\\
\vspace{11pt}
NE 255 - Numerical Simulation in Radiation Transport\\
University of California, Berkeley\\
Department of Nuclear Engineering\\
\vspace{11pt}
Daniel Hellfeld\\
\vspace{11pt}
\today \\
\end{centering}

\vspace{22pt}


The kickoff meeting began with a showcase of the UC Berkeley-led Nuclear Science and Security Consortium (NSSC) winning the 2016 re-competition for the National Nuclear Security Administration (NNSA) Nonproliferation Program (NA-22) grant for \$25M over five years. The purpose of the grant is to train the next generation of nuclear security experts and integrate students in the nuclear field into the national laboratory workforce. In the original 2011-2016 award cycle, the consortium supported almost 160 students, of which about 30\% converted into national laboratory positions (including post-docs and staff). This was very encouraging for me, because my goal upon graduating is to continue my work as a post-doc at a national laboratory. The consortium has changed slightly since the original grant was awarded in 2011. First, the collaboration expanded to include George Washington University, Texas A\&M University, University of Tennessee at Knoxville, and Oak Ridge National Laboratory. Second, the organization of the focus area pillars was restructured slightly and cross-cutting areas were introduced. \\


While there were talks from Los Alamos and Sandia National Laboratories, the meeting was primarily focused on the collaboration between the consortium universities and Lawrence Livermore National Laboratory (LLNL). Presentations were given on a variety of past, present, and future projects in nuclear physics, particle physics, nuclear forensics, and nuclear chemistry. A poster session was held afterwards so students and faculty could engage directly with LLNL scientists and discuss their research. It served as a great opportunity for first and second year graduate students to learn more about the work that is being done at the lab, network with lab scientists, and to potentially find a dissertation project. I specifically had very interesting discussions including metallic magnetic calorimeters for super high-resolution low-energy gamma ray spectroscopy and the challenges associated with scaling the LUX dark matter detector (a two phase liquid-gas xenon TPC) to the next generation detector, LZ.\\

%The deputy directory of NA-22, LTC Ben Miller, gave a short talk about the mission areas of NA-22 and the role NSSC plays in completing those missions. 

I believe the relevant mission areas for NSSC within NA-22 are in the Office of Proliferation Detection which include Nuclear Weaponization and Material Production Detection, Nuclear Weapons and Material Security, and Nonproliferation Enabling Capabilities. Neutral particle transport (both the fundamental concepts and the applications in deterministic and Monte Carlo codes) is crucial in all three of these mission areas. Developments in neutral particle transport will aid in the understanding of the underlying physics in nuclear weapons as well as the ability to more effectively model and simulate these processes. This will progress the designs for both software and hardware tools used to detect special nuclear material and to deter nuclear weapons production. The NSSC plays a critical role in this effort by connecting students (like me!) with lab scientists to participate in cutting-edge research in this field and hopefully contribute to the mission areas of NA-22. 





\end{document}

