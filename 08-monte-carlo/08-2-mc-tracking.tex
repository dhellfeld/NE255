%%%%%%%%%%%%%%%%%%%%%%%%%%%%%%%%%%%%%%%%%%%%%%%%%%%%%%%%%%%%
%%  NE 255
%%

\documentclass[xcolor=x11names,compress]{beamer}

\definecolor{CoolBlack}{rgb}{0.0, 0.18, 0.39}
%% General document %%%%%%%%%%%%%%%%%%%%%%%%%%%%%%%%%%
\usepackage{graphicx}
\usepackage{tikz}
\usetikzlibrary{decorations.fractals}
\usepackage{hyperref}
%%%%%%%%%%%%%%%%%%%%%%%%%%%%%%%%%%%%%%%%%%%%%%%%%%%%%%

%% Beamer Layout %%%%%%%%%%%%%%%%%%%%%%%%%%%%%%%%%%
\useoutertheme[subsection=false,shadow]{miniframes}
\useinnertheme{default}
\usefonttheme{serif}
\usepackage{palatino}
\usepackage{tabu}
\usepackage[normalem]{ulem}
% Links
\usepackage{hyperref}
\definecolor{links}{HTML}{003262}
\hypersetup{colorlinks,linkcolor=,urlcolor=links}

% addition of color
\usepackage{xcolor}
\definecolor{CoolBlack}{rgb}{0.0, 0.18, 0.39}
\definecolor{byellow}{rgb}{0.55037, 0.38821, 0.06142}
\definecolor{dgreen}{rgb}{0.,0.6,0.}
\definecolor{RawSienna}{cmyk}{0,0.72,1,0.45}
\definecolor{forestgreen(web)}{rgb}{0.13, 0.55, 0.13}
\definecolor{cardinal}{rgb}{0.77, 0.12, 0.23}

\setbeamerfont{title like}{shape=\scshape}
\setbeamerfont{frametitle}{shape=\scshape}

\setbeamercolor*{lower separation line head}{bg=CoolBlack}
\setbeamercolor*{normal text}{fg=black,bg=white}
\setbeamercolor*{alerted text}{fg=dgreen} % just testing; I think this looks better
\setbeamercolor*{example text}{fg=black}
\setbeamercolor*{structure}{fg=black}

\setbeamercolor*{palette tertiary}{fg=black,bg=black!10}
\setbeamercolor*{palette quaternary}{fg=black,bg=black!10}

% Margins
\usepackage{changepage}

\mode<presentation>
{
  \definecolor{berkeleyblue}{HTML}{003262}
  \definecolor{berkeleygold}{HTML}{FDB515}
  \usetheme{Boadilla}      % or try Darmstadt, Madrid, Warsaw, Boadilla...
  %\usecolortheme{dove} % or try albatross, beaver, crane, ...
  \setbeamercolor{structure}{fg=berkeleyblue,bg=berkeleygold}
  \setbeamercolor{palette primary}{bg=berkeleyblue,fg=white} % changed this
  \setbeamercolor{palette secondary}{fg=berkeleyblue,bg=berkeleygold} % changed this
  \setbeamercolor{palette tertiary}{bg=berkeleyblue,fg=white} % changed this
  \usefonttheme{structurebold}  % or try serif, structurebold, ...
  \useinnertheme{circles}
  \setbeamertemplate{navigation symbols}{}
  \setbeamertemplate{caption}[numbered]
  \usebackgroundtemplate{}
}
%---

\renewcommand{\(}{\begin{columns}}
\renewcommand{\)}{\end{columns}}
\newcommand{\<}[1]{\begin{column}{#1}}
\renewcommand{\>}{\end{column}}

% adding slide numbers
\addtobeamertemplate{navigation symbols}{}{%
    \usebeamerfont{footline}%
    \usebeamercolor[fg]{footline}%
    \hspace{1em}%
    \insertframenumber/\inserttotalframenumber
}

% equation stuff
\newcommand{\Macro}{\ensuremath{\Sigma}}
\newcommand{\Sn}{\ensuremath{S_N} }
\newcommand{\vOmega}{\ensuremath{\hat{\Omega}}}
\usepackage{mathrsfs}
\usepackage[mathcal]{euscript}
\usepackage{amssymb}
\usepackage{amsthm}
\usepackage{epsfig}
\usepackage{amsmath}
%%%%%%%%%%%%%%%%%%%%%%%%%%%%%%%%%%%%%%%%%%%%%%%%%%
% title stuff for footer
\title{NE 255}
\author{R.\ N.\ Slaybaugh} 
\date{November 22, 2016}

\begin{document}

%%%%%%%%%%%%%%%%%%%%%%%%%%%%%%%%%%%%%%%%%%%%%%%%%%%%%%
%%%%%%%%%%%%%%%%%%%%%%%%%%%%%%%%%%%%%%%%%%%%%%%%%%%%%%
\begin{frame}
\title{NE 255\\Numerical Simulations in Radiation Transport}
\subtitle{Geometry, Collisions, and Scoring}
\titlepage
\end{frame}

%%%%%%%%%%%%%%%%%%%%%%%%%%%%%%%%%%%%%%%%%%%%%%%%%%%%%%
%%%%%%%%%%%%%%%%%%%%%%%%%%%%%%%%%%%%%%%%%%%%%%%%%%%%%%
\begin{frame}{Major Components of MC Algorithm}

\begin{itemize}
  \item \textbf{PDFs}: the physical/mathematical system must be described by a set of pdfs.
  \item \textbf{Random number generator}: a source of random \#s uniformly distributed on the unit interval.
  \item \textbf{Sampling rule}: prescription for sampling the pdf (given having random \#s)
  \item \textit{\textbf{Scoring}: the outcomes must be accumulated/\underline{tallied} for quantities of interest}
  \item \textbf{Error estimation}: an estimate of the statistical error (\underline{variance}) of the solution
  \item \textbf{Variance Reduction}: methods for reducing the variance and computation time simultaneously
  \item \textbf{Parallelization}: efficient use of computers
\end{itemize}
\end{frame}


%%%%%%%%%%%%%%%%%%%%%%%%%%%%%%%%%%%%%%%%%%%%%%%%%%%%%%
%%%%%%%%%%%%%%%%%%%%%%%%%%%%%%%%%%%%%%%%%%%%%%%%%%%%%%
\begin{frame}{Outline}

    \begin{enumerate}
    \item Determining next event location
      \begin{itemize}
      \item Sampling flight path
      \item Distance to boundary %in analytic geometry
      \item Next event selection
%      \item Next event processing
      \end{itemize}
    \item Collision Physics
      \begin{itemize}
      \item Sampling target nuclide
      \item Sampling reaction type
%      \item Sampling exit energy
      \item Sampling exit direction
      \item Sampling exiting particles
      \end{itemize}
    \item Scoring
    \end{enumerate}

\vspace*{1em}
Notes derived from Jasmina Vujic and Paul Wilson
\end{frame}



%%%%%%%%%%%%%%%%%%%%%%%%%%%%%%%%%%%%%%%%%%%%%%%%%%%%%%
%%%%%%%%%%%%%%%%%%%%%%%%%%%%%%%%%%%%%%%%%%%%%%%%%%%%%%
\begin{frame}{Learning Objectives}

    \begin{enumerate}
    \item Understand basic tracking of particles through a geometry
      \begin{itemize}
      \item Understand the steps necessary for tracking particles
      \item Understand the use of mean free path
      \item Sample the distance to the next physics event
      %\item Compute the distance between a point and surface along a given direction
      \item Determine next event
      %\item Process outcome of next event
      \end{itemize}
    \item Understand what sampling needs to happen after a collision
    \item Understand how to translate interactions into a score
    \end{enumerate}

\end{frame}


%%%%%%%%%%%%%%%%%%%%%%%%%%%%%%%%%%%%%%%%%%%%%%%%%%%%%%
%%%%%%%%%%%%%%%%%%%%%%%%%%%%%%%%%%%%%%%%%%%%%%%%%%%%%%
\begin{frame}{Monte Carlo for Transport}

  	\begin{figure}
  	\begin{center}
  		\includegraphics[height=3in,clip]{fig/MC-algorithm}
	\end{center}
  	\end{figure}
  	% we've covered some of the basics that surround this algorithm,
  	% today we'll start covering some of the nuts and bolts
  	% Next time we'll cover tallies.
  	
\end{frame}


%%%%%%%%%%%%%%%%%%%%%%%%%%%%%%%%%%%%%%%%%%%%%%%%%%%%%%
%%%%%%%%%%%%%%%%%%%%%%%%%%%%%%%%%%%%%%%%%%%%%%%%%%%%%%
\begin{frame}{Possible Futures for a Particle}

After we've gotten to \alert{Circle B}, we have a neutral particle:
\begin{itemize}
  \item At point $(x_p , y_p , z_p)$
  \item Moving in direction $(u, v, w)$
  \item With energy $E$
\end{itemize}
What are possible next events?
\pause
  	\begin{figure}
  	\begin{center}
  		\includegraphics<2>[height=1.25in,clip]{fig/collision}
  		\includegraphics<3>[height=1.25in,clip]{fig/boundary-xing}
  		\caption{\only<2>{Collision}\only<3>{fig/Surface Crossing}}
	\end{center}
  	\end{figure}

\end{frame}


%%%%%%%%%%%%%%%%%%%%%%%%%%%%%%%%%%%%%%%%%%%%%%%%%%%%%%
%%%%%%%%%%%%%%%%%%%%%%%%%%%%%%%%%%%%%%%%%%%%%%%%%%%%%%
\begin{frame}{Sampling Distance to Collision}

Collisions are probabilistic
\begin{itemize}
  \item Note that $\Sigma_t$, the total macroscopic cross section, will be a function of space if we have multiple materials
  \item Along a particular path, the \textit{probability of a collision at distance $s$} from the start:
    \begin{align*}
    p_c(s)ds &= \Sigma_t(s) e^{-\Sigma_t(s) s} ds \\
    P_c(s) &= \int_0^s \Sigma_t(s) e^{-\Sigma_t(s) s'}ds' = -e^{-\Sigma_t(s) s'} |_0^s = 1 - e^{-\Sigma_t(s) s}
  \end{align*}
  %\item This is the probability of interaction per unit distance $\times$ probability of traveling $s$ without interacting
%  \vspace*{0.5 em}
%  \pause
  \item The cross section, $\Sigma_t(s)$, is piecewise constant, \underline{but changing}
%  \item Integrating along path: CDF is piecewise
\end{itemize}
%  	\begin{figure}
%  	\begin{center}
%  		\includegraphics[height=.33in,clip]{fig/xsec-cdf}
%	\end{center}
%  	\end{figure}

\end{frame}


%%%%%%%%%%%%%%%%%%%%%%%%%%%%%%%%%%%%%%%%%%%%%%%%%%%%%%
%%%%%%%%%%%%%%%%%%%%%%%%%%%%%%%%%%%%%%%%%%%%%%%%%%%%%%
\begin{frame}{Sampling Distance to Collision}

\begin{itemize}
  \item Variable transformation: measure distance in units of \textit{mean free path}:
  \[n = \Sigma_t(s)s\:,\quad dn = \Sigma_t(s)ds\]
  \item We'll start with the PDF and integrate to get the CDF
 \begin{align*}
    p_c(n)dn &= e^{-n} dn\\
    P_c(n) &= \int_0^n e^{-n'}dn' = -e^{-n'} |_0^n = 1 - e^{-n}
  \end{align*}
  \item Importantly, this is now \underline{independent of the material}
\end{itemize} 

\end{frame}


%%%%%%%%%%%%%%%%%%%%%%%%%%%%%%%%%%%%%%%%%%%%%%%%%%%%%%
%%%%%%%%%%%%%%%%%%%%%%%%%%%%%%%%%%%%%%%%%%%%%%%%%%%%%%
\begin{frame}{Sampling Distance to Collision}

Randomly sample to determine number of mean free paths until next collision, $n_c$

\begin{itemize}
  \item $g(n_c) dn_c = e^{-n_c} dn_c$ 
  \vspace{.5em}
  \item $G(n_c) dn_c = 1 - e^{-n_c}$ 
  \vspace{.5em}
  \item Directly invert to get: $\boxed{n_c = - \ln(1 - \xi)}$ \\
   \hspace*{1.5em} [note $(1-\xi)$ is equivalent to $\xi$]
  \vspace{.5em}
  \item In the absence of material boundaries ($\Sigma_t \neq f(s)$), the distance to a collision, $s_c$, is
  \[s_c = \frac{n_c}{\Sigma_t}\]
\end{itemize}

\end{frame}


%%%%%%%%%%%%%%%%%%%%%%%%%%%%%%%%%%%%%%%%%%%%%%%%%%%%%%
%%%%%%%%%%%%%%%%%%%%%%%%%%%%%%%%%%%%%%%%%%%%%%%%%%%%%%
\begin{frame}{Calculating Distance to Boundary}

\begin{itemize}
  \item Usually have \textit{more than one material}
  \vspace*{1 em}
  \item Distance to boundary is deterministic
  \vspace*{1 em}
  \item Algebra to determine distance between point and surface, $s_b$
  \vspace*{1 em}
  \item Convert it to units of mean free path for the current cell's material, 
  \[n_b = s_b \Sigma_t\] 
\end{itemize}

\end{frame}


%%%%%%%%%%%%%%%%%%%%%%%%%%%%%%%%%%%%%%%%%%%%%%%%%%%%%%
%%%%%%%%%%%%%%%%%%%%%%%%%%%%%%%%%%%%%%%%%%%%%%%%%%%%%%
\begin{frame}{Geometry Representations}

\begin{itemize}
  \item Combinatorial Surfaces
  \begin{itemize}
    \item Define surfaces
    \item Boolean operations combine surfaces to create cells
  \end{itemize}
  \vspace*{1 em}
  \item Combinatorial Solids
  \begin{itemize}
    \item Choose solid objects
    \item Boolean operations combine objects to create regions
  \end{itemize}
  \vspace*{1 em}
  \item B-Rep (Vertex-Edge)
  \begin{itemize}
    \item Each object is a single set of vertices and edges connecting them
  \end{itemize}
\end{itemize}
\vspace*{1 em}
We're skipping how to find $s_b$, just know that we can find it using the internal geometry representation

\end{frame}


%%%%%%%%%%%%%%%%%%%%%%%%%%%%%%%%%%%%%%%%%%%%%%%%%%%%%%
%%%%%%%%%%%%%%%%%%%%%%%%%%%%%%%%%%%%%%%%%%%%%%%%%%%%%%
\begin{frame}{Option A: Collision}

  \underline{$n_b > n_c$}:
  \begin{itemize}
    \item Boundary is further away than collision
    \item \alert{Collision occurs}
  \end{itemize}
    \vspace*{0.5 em}
  \pause

\begin{itemize}
  \item Using physics models and/or cross-sections
  \begin{itemize}
    \item Sample target nuclide
    \item Sample reaction type
    \item Sample new direction 
    \item Sample new energy 
    \item Sample exiting particles 
  \end{itemize}
  \item Some of these may depend on one another
  \vspace*{0.5 em}
  \pause
  \item Repeat
  \begin{itemize}
    \item Sample new $n_c$ following collision
    \item Calculate new $n_b$ in new direction
  \end{itemize}
\end{itemize}


\end{frame}

%%%%%%%%%%%%%%%%%%%%%%%%%%%%%%%%%%%%%%%%%%%%%%%%%%%%%%
%%%%%%%%%%%%%%%%%%%%%%%%%%%%%%%%%%%%%%%%%%%%%%%%%%%%%%
\begin{frame}{Option B: Cell Boundary}

  \underline{$n_b < n_c$}:
  \begin{itemize}
    \item Boundary is closer than collision
    \item \alert{Boundary crossing occurs}
  \end{itemize}
    \vspace*{0.5 em}
  \pause

  \begin{itemize}
  \item Move particle along ray
  \begin{itemize}
    \item Update $n_c = n_c - n_b$
  \end{itemize}
  \item \textbf{DO NOT SAMPLE} for new $n_c$
  \vspace*{1 em}
  \pause
  \item Calculate new $n_b$ in new cell
  \begin{itemize}
    \item New set of boundaries
    \item New value of $\Sigma_t$
  \end{itemize}
\end{itemize}

\end{frame}


%%%%%%%%%%%%%%%%%%%%%%%%%%%%%%%%%%%%%%%%%%%%%%%%%%%%%%
%%%%%%%%%%%%%%%%%%%%%%%%%%%%%%%%%%%%%%%%%%%%%%%%%%%%%%
\begin{frame}{So You Had a Collision?}

\begin{itemize}
  \item Sample \textbf{target nuclide} for a mixture with $J$ nuclides
    \[\Sigma_t = \sum_{j=1}^J N_j \sigma_{t,j}\]
  \item \textit{Discrete PDF} to determine which nuclide is hit
    \[p_i = \frac{\Sigma_{t,j}}{\Sigma_t}\]
  \pause
  \item Sample \textbf{reaction type} for an nuclide with R types of reactions
     \[\Sigma_{t,j} = \sum_{x=1}^R \Sigma_{x,j}\]
  \item \textit{Discrete PDF} to determine which reaction occurs
    \[p_x = \frac{\Sigma_{x,j}}{\Sigma_{t,j}}\]
\end{itemize}

\end{frame}


%%%%%%%%%%%%%%%%%%%%%%%%%%%%%%%%%%%%%%%%%%%%%%%%%%%%%%
%%%%%%%%%%%%%%%%%%%%%%%%%%%%%%%%%%%%%%%%%%%%%%%%%%%%%%
\begin{frame}{Outcome of Reaction}

\begin{itemize}
    \item Particle maybe absorbed
\vspace*{1em}
    \item Particle may continue its history in a \textit{different direction} and/or with a \textit{different energy}
\vspace*{1em}
    \item Energy-angle distributions are tabulated in different formats
    \begin{itemize}
      \item Scattering laws have analytic forms with parameters in data tables\\
      (Direct inversion or rejection sampling)
      \item Tabulated data that describes a piecewise analytic interpolation\\
      (Hybrid sampling; we skipped this)
    \end{itemize}
\end{itemize}

\end{frame}


%%%%%%%%%%%%%%%%%%%%%%%%%%%%%%%%%%%%%%%%%%%%%%%%%%%%%%
%%%%%%%%%%%%%%%%%%%%%%%%%%%%%%%%%%%%%%%%%%%%%%%%%%%%%%
\begin{frame}{Using a Scattering Angle}

Scattering angles are defined relative to the original direction (considered as the z-axis)

\begin{itemize}
    \item Polar angle, $\phi$, determined by sampling from data
\vspace*{0.5em}
    \item Azimuthal angle, $\theta$, determined by sampling isotropically
    \vspace*{0.5em}
   % \item Consider old/original direction as the z-axis in a different coordinate system
    \item The new direction is $\bigl(\sin(\phi) \cos(\theta), \sin(\phi) \sin(\theta), \cos(\theta)\bigr)$ \[= \bigl(\sqrt{1 - \mu^2} \cos(\theta),  \sqrt{1 - \mu^2} \sin(\theta), \mu \bigr)\]
\end{itemize}

  	\begin{figure}
  	\begin{center}
  		\includegraphics[height=1in,clip]{fig/scattering-angle-2}
	\end{center}
  	\end{figure}

\end{frame}


%%%%%%%%%%%%%%%%%%%%%%%%%%%%%%%%%%%%%%%%%%%%%%%%%%%%%%
%%%%%%%%%%%%%%%%%%%%%%%%%%%%%%%%%%%%%%%%%%%%%%%%%%%%%%
\begin{frame}{Summary of Part I}

We've developed a general sense of using MC for neutron transport
\begin{itemize}
    \item Basic Algorithm
    \pause
    \vspace*{0.5 em}
    \item We can determine if particles have collisions or cross boundaries
    \item \textit{After a collisions} we need to determine many things associated with the collisions (target, reaction, direction, energy)
    \item Repeat analysis for collisions/crossing until particle \textbf{terminates}
\end{itemize}


\end{frame}



\end{document}
