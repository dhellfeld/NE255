\documentclass[12pt]{article}
\usepackage[top=1in, bottom=1in, left=1in, right=1in]{geometry}

\usepackage{setspace}
\onehalfspacing

\usepackage{amssymb}
%% The amsthm package provides extended theorem environments
\usepackage{amsthm}
\usepackage{epsfig}
\usepackage{times}
\renewcommand{\ttdefault}{cmtt}
\usepackage{amsmath}
\usepackage{graphicx} % for graphics files
\usepackage{tabu}

% Draw figures yourself
\usepackage{tikz} 

% writing elements
\usepackage{mhchem}

\usepackage{paralist}

% The float package HAS to load before hyperref
\usepackage{float} % for psuedocode formatting
\usepackage{xspace}

% from Denovo Methods Manual
\usepackage{mathrsfs}
\usepackage[mathcal]{euscript}
\usepackage{color}
\usepackage{array}

\usepackage[pdftex]{hyperref}
\usepackage[parfill]{parskip}

% math syntax
\newcommand{\nth}{n\ensuremath{^{\text{th}}} }
\newcommand{\ve}[1]{\ensuremath{\mathbf{#1}}}
\newcommand{\Macro}{\ensuremath{\Sigma}}
\newcommand{\rvec}{\ensuremath{\vec{r}}}
\newcommand{\vecr}{\ensuremath{\vec{r}}}
\newcommand{\omvec}{\ensuremath{\hat{\Omega}}}
\newcommand{\vOmega}{\ensuremath{\hat{\Omega}}}
\newcommand{\even}{\ensuremath{\phi^g}}
\newcommand{\odd}{\ensuremath{\vartheta^g}}
\newcommand{\evenp}{\ensuremath{\phi^{g'}}}
\newcommand{\oddp}{\ensuremath{\vartheta^{g'}}}
\newcommand{\Sn}{\ensuremath{S_N} }
\newcommand{\Ye}[2]{\ensuremath{Y^e_{#1}(\vOmega_#2)}}
\newcommand{\sigg}[1]{\ensuremath{\Macro^{gg'}_{s\,#1}}}
\newcommand{\psig}{\ensuremath{\psi^g}}
%---------------------------------------------------------------------------
%---------------------------------------------------------------------------
\begin{document}
\begin{center}
{\bf NE 255, Fa16 \\
Equation Discretization\\
October 6, 2016}
\end{center}

\setlength{\unitlength}{1in}
\begin{picture}(6,.1) 
\put(0,0) {\line(1,0){6.25}}         
\end{picture}

So far we've dealt with
\begin{compactitem}
\item Discretization of \textit{time} using finite difference method (Taylor expand points and combine)
\item Discretization of \textit{energy} using the multigroup approximation, where we assume group-integrated values. 
\item Expanding sources, in particular scattering, in spherical harmonics, which we can reduce to Legendre Polynomials in the case of azimuthal symmetry.
\item Discretization of \textit{angle} using either 
  \begin{compactitem}
  \item $S_N$: get solutions along specific angle sets (quadrature points), use corresponding quadrature weights to integrate over angle
  \item $P_N$: expand the angular flux in spherical harmonics, which we only do in 1-D so Legendre polynomials in practice, and solve a set of coupled equations for each expansion term (with closure relations at $n=0 and n=N+1$).
  \item $SP_N$: WORDS
  \end{compactitem}
\end{compactitem}

When we do all of this we get $t=0,\dots, T$ equations in time, $g=0,\dots,G$ equations in energy, and a number of equations that depends on which angular approach we take. However, we still have one major item to deal with...

\section*{Space}


\end{document}