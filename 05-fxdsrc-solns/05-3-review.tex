\documentclass[12pt]{article}
\usepackage[top=1in, bottom=1in, left=1in, right=1in]{geometry}

\usepackage{setspace}
\onehalfspacing

\usepackage{amssymb}
%% The amsthm package provides extended theorem environments
\usepackage{amsthm}
\usepackage{epsfig}
\usepackage{times}
\renewcommand{\ttdefault}{cmtt}
\usepackage{amsmath}
\usepackage{graphicx} % for graphics files
\usepackage{tabu}

% Draw figures yourself
\usepackage{tikz} 

% writing elements
%\usepackage{mhchem}

\usepackage{paralist}

% The float package HAS to load before hyperref
\usepackage{float} % for psuedocode formatting
\usepackage{xspace}

% from Denovo Methods Manual
\usepackage{mathrsfs}
\usepackage[mathcal]{euscript}
\usepackage{color}
\usepackage{array}

\usepackage[pdftex]{hyperref}
\usepackage[parfill]{parskip}

% math syntax
\newcommand{\nth}{n\ensuremath{^{\text{th}}} }
\newcommand{\ve}[1]{\ensuremath{\mathbf{#1}}}
\newcommand{\Macro}{\ensuremath{\Sigma}}
\newcommand{\rvec}{\ensuremath{\vec{r}}}
\newcommand{\vecr}{\ensuremath{\vec{r}}}
\newcommand{\omvec}{\ensuremath{\hat{\Omega}}}
\newcommand{\vOmega}{\ensuremath{\hat{\Omega}}}
\newcommand{\even}{\ensuremath{\phi^g}}
\newcommand{\odd}{\ensuremath{\vartheta^g}}
\newcommand{\evenp}{\ensuremath{\phi^{g'}}}
\newcommand{\oddp}{\ensuremath{\vartheta^{g'}}}
\newcommand{\Sn}{\ensuremath{S_N} }
\newcommand{\Ye}[2]{\ensuremath{Y^e_{#1}(\vOmega_#2)}}
\newcommand{\Yo}[2]{\ensuremath{Y^o_{#1}(\vOmega_#2)}}
\newcommand{\sigg}[1]{\ensuremath{\Macro^{gg'}_{s\,#1}}}
\newcommand{\psig}{\ensuremath{\psi^g}}
\newcommand{\Di}{\ensuremath{\Delta_i}}
\newcommand{\Dj}{\ensuremath{\Delta_j}}
\newcommand{\Dk}{\ensuremath{\Delta_k}}
%---------------------------------------------------------------------------
%---------------------------------------------------------------------------
\begin{document}
\begin{center}
{\bf NE 255, Fa16 \\
Source Iteration Redux\\
Nov 15, 2016}
\end{center}

\setlength{\unitlength}{1in}
\begin{picture}(6,.1) 
\put(0,0) {\line(1,0){6.25}}         
\end{picture}

Let's revisit ``source iteration" and how that works to get a better handle on within group iteration. Let's assume
\begin{compactitem}
\item 1D
\item azimuthally symmetric
\item steady state
%\item diamond difference spatial differencing ($i = 1, \dots, I$)
\item $S_N$ angular discretization ($a = 1, \dots, N(N+2)$; $N(N+2)=M$)
\item $P_N$ with $N=1$: linearly anisotropic scattering ($l = 0, 1$ so $\Sigma_s = \Sigma_{s0} + 3\mu_a \mu_{a'} \Sigma_{s1}$)
\item multigroup in energy ($g = 1, \dots, G$)
\end{compactitem}

With that, we have
\begin{align}
%\mu_a \frac{d}{dx}\psi_{a,i}^g + \Sigma_{t,i} \psi_{a,i}^g &= 2\pi \sum_{g'=1}^G
%  \sum_{l=0}^N \Sigma_{i,sl}^{gg'} \tilde{\psi^{g'}_i} + q_{a,i}^g\\
\mu_a \frac{d}{dx}\psi_{a}^g(x) + \Sigma_{t}(x) \psi_{a}^g(x) &= 2\pi \sum_{g'=1}^G
  \sum_{l=0}^N \Sigma_{sl}^{gg'}(x) \tilde{\psi^g_l}(x) + q_{a}^g(x)\\
  %
\tilde{\psi^g_0}(x) &= \sum_{a=1}^M w_a \psi_a^g(x)\:; \quad P_0 = 1\\
\tilde{\psi^g_1}(x) &= \sum_{a=1}^M w_a \mu_a \psi_a^g(x)\:; \quad P_1 = \mu
\end{align}
%
We've 
\begin{compactitem}
\item converted the angular integration in the source term to a quadrature integration ($\sum_{a=1}^M w_a f_a$)
\item expanded the double differential scattering cross section in Legendre polynomials ($\sum_{l=0}^N P_l f_l$)
\item applied groups ($x^g$)
\end{compactitem} 

Next, let's add the iteration index for source iteration.  
\begin{align}
\mu_a \frac{d}{dx}\psi_{a}^{g, (k+1)}(x) + \Sigma_{t}(x) \psi_{a}^{g, (k+1)}(x) &= 2\pi \sum_{g'=1}^G
  \sum_{l=0}^N \Sigma_{sl}^{gg'}(x) \tilde{\psi^{g', (k)}_l}(x) + q_{a}^g(x)\\
  %
\tilde{\psi}^{g, (k)}_0(x) &= \sum_{a=1}^M w_a \psi_a^{g, (k)}(x)\:; \quad P_0 = 1\\
\tilde{\psi}^{g, (k)}_1(x) &= \sum_{a=1}^M w_a \mu_a \psi_a^{g, (k)}(x)\:; \quad P_1 = \mu
\end{align}

You can see that because the \textit{source} depends on the flux from angles and energies other than the one we are in, we need to \textit{iterate}. Hence, \textbf{source iteration}. We would still need to iterate with only one group because of the angle component. If we didn't have scattering, then we wouldn't need to iterate. 

To add spatial discretization, define the right hand side as $s_i$:
\begin{align}
\psi_{i,a}^{g,(k+1)} &= \frac{s_{i}^{g,(k)} + \frac{2}{(1\pm\alpha_i)}\frac{|\mu|}{\Di}\bar{\psi}^{g,(k+1)}_{i\mp1/2}}{
      \Sigma_{i} + \frac{2}{(1\pm\alpha_i)}\frac{|\mu|}{\Di}} \label{eq:dd}\\
%
%\psi_{i}^{g,(k+1)} &= \frac{1}{2}\bigl((1+\alpha_i)\psi_{i+1/2}^{g,(k+1)}+(1-\alpha_i)\psi_{i-1/2}^{g,(k+1)}\bigr)\\
%
\psi_{i\pm1/2}^{g,(k+1)} &= \frac{2}{(1\pm\alpha_i)}\psi_{i}^{g,(k+1)}-
    \frac{(1\mp\alpha_i)}{(1\pm\alpha_i)}\bar{\psi}_{i\mp1/2}^{g,(k+1)} \label{eq:out}
\end{align}
where the overbar indicates incoming flux (based on direction).

With all of this, we can see the iteration levels. We have an initial guess $\psi^{(0)}$ for each group, angle, and cell. For a given group, do the following \underline{until convergence of the flux in that group}:
\begin{enumerate}
\item For each angle
    \begin{enumerate}
    \item Start at the appropriate cell where we have known incoming flux moments, $\bar{\psi}_{i\mp1/2}^{g}$
    \item Calculate the source guess in that cell using the past flux value (because it is known at all angles and energies):
    \[
    s_{i}^{g,(k)} = 2\pi \sum_{g'=1}^G
  \sum_{l=0}^N \Sigma_{sl}^{gg'}(x) \tilde{\psi^g_l}(x) + q_{a}^g(x)
    \]
    \item Calculate and store the updated flux in that cell, $\psi_{i,a}^{g,(k+1)}$, using \autoref{eq:dd} 
    \item Calculate the updated outgoing flux in that cell, $\psi_{i\pm1/2}^{g,(k+1)}$, using \autoref{eq:out}
    \item Set the incoming flux in the next cell to be the outgoing flux in the previous cell:
    \[
    \bar{\psi}_{i\mp3/2}^{g} = \psi_{i\pm1/2}^{g,(k+1)}
    \]
    \item move to the next cell
  \end{enumerate}
  \item move to the next angle
\end{enumerate}


\end{document}
